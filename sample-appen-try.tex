%!TEX ROOT = thesis.tex
\chapter{Appendix 2: What is appendix}

Appendix is included in your report as it is information that is not essential to explain your findings, but that supports your analysis (especially repetitive or lengthy information), validates your conclusions or pursues a related point should be placed in an appendix (plural appendices). Sometimes excerpts from this supporting information (i.e. part of the data set) will be placed in the body of the report but the complete set of information ( i.e. all of the data set) will be included in the appendix. Examples of information that could be included in an appendix include figures/tables/charts/graphs of results, statistics, questionnaires, transcripts of interviews, pictures, lengthy derivations of equations, maps, drawings, letters, specification or data sheets, computer program information.

There is no limit to what can be placed in the appendix providing it is relevant and reference is made to it in the report. The appendix is not a catch net for all the semi-interesting or related information you have gathered through your research for your report: the information included in the appendix must bear directly relate to the research problem or the report's purpose. It must be a useful tool for the reader 